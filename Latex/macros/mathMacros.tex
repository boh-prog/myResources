% Common math math
\usepackage{bigints}  % makes big integrals 
\everymath{\displaystyle} 
\newcommand{\B}{\left}	  
\newcommand{\E}{\right}
\newcommand{\Acronym}[2]{\B( #1_{_{#2}}\E)}
\newcommand{\Avg}[1]{\langle#1\rangle}
\newcommand{\AvgBig}[1]{\B\langle#1\E\rangle}
\newcommand{\Parenthesis}[1]{\B(#1\E)}
\newcommand{\SqBracket}[1]{\B[#1\E]}
\newcommand{\CurlyBracket}[1]{\B\{#1\E\}}
\newcommand{\Sub}[2]{#1_{_{#2}}} 
\newcommand{\Up}[2]{#1^{^{#2}}} 
\newcommand{\UpSub}[3]{#1_{_{#2}}^{^{#3}}}
\newcommand{\Result}[1]{\bm{\boxed{#1}}}
\newcommand{\Limit}[2]{\underset{#1\rightarrow#2}{\lim}} 
\newcommand{\Function}[2]{#1_{\Parenthesis{#2}}} 
\newcommand{\SubFunction}[3]{#1_{_{#2}{(#3)}}}
\newcommand{\Units}[1]{\,\SqBracket{#1}} 
\newcommand{\UnitsFrac}[2]{\,\SqBracket{\frac{#1}{#2}}} 
\newcommand{\Abs}[1]{\B\lvert#1\E\rvert} 
\newcommand{\Norm}[1]{\B\lVert#1\E\rVert} 

% General 
\newcommand{\EqRef}[1]{Eq.\eqref{#1}}
\newcommand{\FigRef}[1]{Fig.\ref{#1}}
\newcommand{\TabRef}[1]{Tab.\ref{#1}}

% Derivatives 
\newcommand{\Partial}[1]{\Sub{\partial}{#1}} 
\newcommand{\PartialFrac}[2]{\frac{\partial #1}{\partial #2}} 
\newcommand{\PartialFracN}[3]{\frac {\Up{\partial}{#3} #1}{\partial \Up{#2}{#3}}} 
\newcommand{\PartialDer}[2]{\frac{\partial #1}{\partial #2}} 
\newcommand{\PartialDerN}[3]{\frac {\Up{\partial}{#3} #1}{\partial \Up{#2}{#3}}} 
\newcommand{\TotalDer}[2]{\frac{d #1}{d #2}} 

% Trig  
\newcommand{\Sin}[1]{\sin\Parenthesis{#1}}
\newcommand{\SinFrac}[2]{\sin\Parenthesis{ \frac{#1}{#2} }}
\newcommand{\Cos}[1]{\cos\Parenthesis{#1}}
\newcommand{\CosFrac}[2]{\cos\Parenthesis{ \frac{#1}{#2} }}
\newcommand{\Tan}[1]{\tan\Parenthesis{#1}}
\newcommand{\TanFrac}[2]{\tan\Parenthesis{ \frac{#1}{#2} }}
\newcommand{\Ln}[1]{\ln\B|#1\E|}
\newcommand{\LnFrac}[2]{\ln\B|\frac{#1}{#2}\E|}

% Symblos 
% Countercloskwise            
\def\Counterclockwise#1{\tikz[baseline=(A.base)]
                        \draw[-stealth,line width=.035em]  
                        (0,0) node[circle, inner sep=0cm](A){$#1$}
                        let \p1=(A.center),\p2=(A.west), \n1={\x1-\x2} in
                        (-90:\n1) arc(-90:150:\n1);}
% Clockwise 
\def\Clockwise#1{\tikz[baseline=(A.base)]
                 \draw[-stealth,line width=.035em]  
                 (0,0) node[circle, inner sep=0cm](A){$#1$}
                 let \p1=(A.center),\p2=(A.east), \n1={\x1-\x2} in
                 (90:\n1) arc(90:-150:\n1);}


